\documentclass[answers, 12pt]{exam}

\usepackage {fontspec}
\setmainfont {Calibri}
\usepackage[letterpaper,margin=1in,top=0.5in,includehead]{geometry}

\usepackage{pgf,pgffor}

\usepackage{array}
\usepackage{tabu}

\usepackage[at]{easylist}
\usepackage{color}
\definecolor{light-gray}{gray}{0.95}

\include{Common}

\usepackage[normalem]{ulem}
\usepackage{multicol}

\usepackage {graphicx}
\graphicspath {{.}}

\usepackage {xcolor, soul}

\usepackage {calc}
\linespread{1.09}
\setlength{\parskip}{0.14in}
\setlength{\parindent}{0pt}
\usepackage{enumitem}
\usepackage{amssymb}

\firstpageheader{\includegraphics[height=0.68in, keepaspectratio]{USC}}{}{}
\runningheader{\includegraphics[height=0.68in, keepaspectratio]{USC}}{}{}
\runningfooter{}{}{}

\extraheadheight{\heightof{\includegraphics[height=0.68in, keepaspectratio]{USC}}}

\begin {document}

{\centering Performance Task on Sampling for Respondents\par}
Group Number: \underline{ 3 } 
Block: \underline{ P }

\newlist{todolist}{itemize}{2}
\setlist[todolist]{leftmargin=1em,label=\(\square\)}

\begin{multicols}{2}[Names:]
  \newcommand\longunderline[1]{\uline{\hspace*{0.1cm}#1\hfill}}
  \authors[\longunderline]
\end {multicols}

\renewcommand{\thequestion}{\Alph{question}}
\unframedsolutions
\renewcommand{\solutiontitle}{\vspace{-0.20in}}
\begin {questions}
\question{Thesis Title {\color{red}(2 points)}}
  \begin{solution}
    \researchtitle
  \end {solution}

\question{Define precisely the target population {\color{red}(2 points)}}
  \begin {solution}
    USC STEM Senior High Students
  \end{solution}

\question{Attach the sampling frame {\color{red}(5 points)}}
\question{Population Size: \underline{ 357 } (1 point)}
\question{Compute for the sample size {\color{red}(3 points)}}
  \begin {solution}
    \[n = \frac{N}{1+Ne^2} = \frac{357}{1 + 357*(0.1)^2} \approx 79\]
  \end {solution}
\question{Suitable Probability Sampling Procedure: Simple Random Sampling
    {\color{red}(2 points)}}
\question{Describe the steps in the procedure. {\color{red}(2 points)}}
  \begin {solution}
    \sethlcolor{light-gray}
    \newcommand{\code}[1]{\hl{\texttt{#1}}}
    We can do the following to simulate a shuffle/lottery:
    \begin{enumerate}
    \item We place the dataset on a single column. e.g. \code{A1:A357}
      
    \item We generate \(N\) numbers using \code{RAND} e.g. \code{B1 = RAND()} and
      autofill down to \code{B357}.

    \item We use \code{RANK} to simulate a shuffle. We only need to generate
      \(n\) numbers. e.g.\\\code{C1 = RANK(B1, B\$1:B\$357)} and autofill down to
      \code{C79}. The \code{\$} means to not increment on autofill.

    \item We autofill the next column with incrementing numbers. e.g. \code{D1 = 1}, \code{D2 = 2}, \ldots

    \item We can sort the random numbers by using \code{SMALL(RANKCOLUMN, INCREMENTINGCELL)}. e.g. \code{E1 = SMALL(C\$1:C\$257, D1)}, and autofill to \code{E257}.

    \item We use \code{INDEX(DATASETCOLUMN, SORTEDCELL, 1)} to select certain values on
      the dataset. e.g. \code{F1 = INDEX(A\$1:A\$357, E1, 1)} and autofill down
      to \code{F79}.

    \end {enumerate}
  \end {solution}
\question{Indiciate the corresponding number of each of your respondents.
    (Attach evidence of the random selection of respondents) {\color{red} (20 points)}}
  \begin {solution}
    \begin{tabu} to 1\textwidth { | X[c] | X[c] | X[c] | }
      \hline
      \rowfont{\vspace{-1.2em} \color{red}}
      \begin{todolist}
      \item Comfortable with the use of a spread sheet to generate the random numbers
      \item Generates the random numbers in less than the given time
      \item Sorts the generated numbers
      \end {todolist} &
      \begin{todolist}
      \item Comfortable with the use of a spread sheet to generate the random numbers
      \item Generates the random numbers in less than the given time
      \item Sorts the generated numbers
      \end {todolist} &
      \begin{todolist}
      \item Comfortable with the use of a spread sheet to generate the random numbers
      \item Generates the random numbers in less than the given time
      \item Sorts the generated numbers
      \end {todolist} \\
      \hline
      \rowfont{\vspace{-0.6em} \color{red}} 15-20 points & 11-15 points & 1-5 points \\ 
      \hline
    \end{tabu}
  \end {solution}
\question{List the specific names of your respondents (use a separate sheet
    for this list)}
\end {questions}

\vspace{2em} {\color{red} Final write-up submitted on time (10 points)}
\pagebreak
\fontsize{8pt}{-1pt}\selectfont
\begin{multicols}{3}
\begin{enumerate}
\foreach \x in \respondents {
  \item \x
}
\end{enumerate}
\end{multicols}
\end {document}
%%% Local Variables: 
%%% TeX-engine: luatex 
%%% End: 
