\documentclass [12pt] {scrreprt}
\usepackage [margin=1in, letterpaper]{geometry}

\usepackage {luatex85}
\usepackage{standalone}

\linespread {1.3}
% \usepackage {fontspec}
% \setmainfont {Tahoma}

\setcounter{tocdepth}{5}
\begin {document}
\tableofcontents
\setlength{\parskip}{1em}
\setlength{\parindent}{0pt}
\documentclass {article}
\usepackage {fontspec}
\setmainfont [ItalicFont={Tahoma Faux Italic}]{Tahoma}
\begin {document}
\include {Common}
\begin {titlepage}
  \centering
	{\huge \researchtitle

  }
  \vfill
  {\LARGE Isang Sulating Pananaliksik na Iniharap kay \teacher ng

    \uppercase{\school}

  }
  \vspace{1cm}
  {\LARGE Bilang Bahagi ng Pagpatupad sa Pangangailangan ng Kursong
    Filipino 11, Pagbasa at Pagsuri ng Iba't ibang Teksto Tungo sa Pananaliksik

  }
  \vfill
  {\Large\itshape \authors

  }
\end {titlepage}
\end {document}
%%% Local Variables: 
%%% TeX-engine: luatex 
%%% End: 
\RedeclareSectionCommand[
  beforeskip=0pt,
  afterskip=0pt]{chapter}
\RedeclareSectionCommand[
  beforeskip=0pt,
  afterskip=1em]{section}
\setkomafont{chapter}{\normalfont\bfseries}
\setkomafont{section}{\normalfont\bfseries}
\pagebreak
\chapter {Ang Suliranin at ang Kaligiran Nito}
\section {Rasyunal}
Ang Pilipinas ay may mayamang kultura. Ito ay isinakop ng mga Kastila, Amerikano
at mga Hapones. Ang isang resulta nito ay pagkakarami ng mga wikain sa bansang
ito. Ang Pilipinas rin ay isang kapuluan at kaya maraming mga dayalektong nabuo
dahil sa pagkakalayo ng mga tao. Layunin ng pagsusuring ito ay malaman kung ano
ang tamang wikang gamitin sa mga piling asignatura.

Isa sa mga problema sa mga estudyante ay ang hindi pagkakaintindi sa leson dahil
sa problema sa wika. Gusto namin itong iayos sa paggawa ng pagsusuring ito.
Layunin namin na lahat na mga estudyante ay makakaintindi sa leson ng kanilang
guro sa paggamit ng wikang wasto para sa mga piling asignatura. Bilang mga
kapwang mag-aaral, ang mga manunuri ay nakakaranas rin ng problemang ito.

Ito rin ay makakatulong sa guro. Hindi na palaging paulit-ulit magturo ang guro
ng isang paksa kapag maigi ng naiintindihan ng kanyang mga mag-aaral.
Masmadaling makakapagpatuloy ang guro sa kanyang mga leson at masmaraming
matutunan ang mga mag-aaral.

\pagebreak
\section {Paglalahad ng Suliranin at Kahalagahan ng Pag-aaral}
Sa Estados Unidos, maraming mga imigrante bata
pumunta mula sa pagiging monolingual sa kanilang wikang sa bilingual sa wikang
iyon at sa Ingles, lamang sa mga end up monolingual mga nagsasalita ng Ingles.
Ang tawag dito ay \emph{subtractive bilingualism}, at ito ay maaaring magkaroon
ng negatibong epekto sa mga bata, na madalas mong ikahiya kanilang wikang.
(Santrock, 1990, p.151)

Being an immigrant is difficult because you face different people and different
language. Immigrant students are often ashamed of their home language and also
find it difficult for them to understand the language. But given enough time,
the language will be learned. (Santrock, p.151)

A student's comprehension relies massively on the language medium being used to
teach. 

Language, The source of our communications. It is essential for us to learn this
for us to communicate or to interact with other people in the classroom,
language is important not only in class, but a social and economic advancement
in our country. Learning the country's common or standard language is  the most
essential for the foreigner's economic and social progress. 

\pagebreak
\section {Saklaw at Limitasyon ng Pag-aaral at Depinisyon ng mga Terminolohiya}
Ang pagsusuring ito ay nalilimita lamang sa mga USC SHS STEM students. Walang
kasiguraduhan na ito ay pareho rin sa ibang mga paaralan.

Nais naming malaman kung anong wika ang tamang gamitin sa mga \emph{piling
  asignatura}. Hindi layunin ng pagsusuring ito ang magkaroon ng iisang wika
lamang sa lahat ng mga asignatura.

Maspinahahalagahan namin ang sagot na nasa choices kaysa sa ``Iba.''

Ang mga respondente ay ipinili gamit ang \emph{Simple Random Sampling}.
\vfill
Subtractive Bilingualism -- Isang paraan ng pagpili na parang lottery
Simple Random Sampling -- Isang paraan ng pagpili na parang lottery
\pagebreak
\end {document}
%%% Local Variables: 
%%% TeX-engine: xetex
%%% End: 