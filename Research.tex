\documentclass [11pt] {report}
\usepackage [margin=1in, left=1.5in, letterpaper]{geometry}

% \usepackage[titles]{tocloft}

\usepackage {titlesec}

\usepackage {luatex85}

\usepackage{standalone}
\usepackage{calc}

\linespread {1.3}
\usepackage {fontspec}
%\setmainfont {Tahoma}
%\setmainfont [ItalicFont={Tahoma Faux Italic}]{Tahoma}

\usepackage{enumitem}
\usepackage{titlesec}

% \renewcommand{\cftchapfont}{%
%   \normalfont\bfseries
% }

\begin {document}
\documentclass[class=scrartcl]{standalone}
\usepackage {standalone}
% \usepackage {fontspec}
% \setmainfont [ItalicFont={Tahoma Faux Italic}]{Tahoma}
\begin {document}
\include {Common}
\begin {titlepage}
  \centering
	{\LARGE\researchtitle

  }
  \vfill
  {\Large Isang Sulating Pananaliksik na Iniharap kay\\
    \teacher\\
    ng\\
    \uppercase{\school}

  }
  \vspace{1cm}
  {\Large Bilang Bahagi ng Pagpatupad sa Pangangailangan ng Kursong
    Filipino 11, Pagbasa at Pagsuri ng Iba't ibang Teksto Tungo sa Pananaliksik

  }
  \vfill
  {\large\itshape \authors

  }
\end {titlepage}
\end {document}
%%% Local Variables: 
%%% TeX-engine: luatex 
%%% End: 
\newcommand\discard[1]{}
% \RedeclareSectionCommand[
%   font=\normalfont\bfseries,
%   tocentrynumberformat=\discard,
%   beforeskip=0pt,
%   innerskip=1pt,
%   afterskip=1sp]{chapter}
% \RedeclareSectionCommand[
%   font=\normalfont\bfseries,
%   tocentrynumberformat=\discard,
%   beforeskip=0pt,
%   afterskip=1sp]{section}
\parindent=0.5in
\parskip=\baselineskip
\titleformat{name=\chapter}[display]
{\bfseries}
{\centering KABANATA \Roman{chapter}}
{-\parskip}{\centering\uppercase}

\titleformat{name=\section}[display]
{\bfseries}
{}
{-\parskip}{}

\titlespacing*{name=\chapter}{0pt}{0pt}{-\baselineskip}
\titlespacing*{name=\section}{0pt}{0pt}{0pt}
%\renewcommand*{\chapterformat}{\centering Kabanata \thechapter:\enskip}
%\renewcommand*{\sectionformat}{}
\pagebreak
\chapter {Ang Suliranin at ang Kaligiran Nito}
\section {Rasyunal}
Sa pagsakop ng mga Kastila, Amerikano at mga Hapones, nagkaroon ng mayamang
kultura ang Pilipinas. Ito ay nagresulta ng pagkakarami ng mga wikain sa
bansa. Ang Pilipinas rin ay isang kapuluan at kaya maraming mga
dayalektong nabuo dahil sa pagkakalayo ng mga tao. Layunin ng pagsusuring ito ay
malaman kung ano ang tamang wikang gamitin sa mga piling asignatura.

Isa sa mga problema ng mga magaaral ay ang hindi pagkakaintindi ng leson dahil
sa kahirapan sa pag-intindi sa wika. Gusto ng mga mananaliksik na ito ay
solusyonan sa pamamagitan ng paggawa ng pagsusuring ito. Pakay namin na
maiintindihan ng lahat na mga estudyante ang mga leksiyong itintuturo ng
kanilang mga guro sa pagpapalitaw kung anu-anong mga wika ang wasto sa mga
piling asignatura. Sa panahon ng pagsulat, ang mga manunuri ay kapwang mag-aaral
at nakakaranas rin ng problemang ito.

Hindi lng ito nakabubuti sa mga estudyante. Ito rin ay makatutunay na
makapakinabangan ng mga guro. Hindi na palaging paulit-ulit magturo ang guro sa
isang paksa kapag maigi nang naiintindihan ng kanyang mga tinuturuan. Ang daloy
ng interaksyon ng guro sa kaniyang mga estudyante ay aayos na siya na ma’y
siguradong ikalulugod ng dalawang panig.

\pagebreak
\section {Paglalahad ng Suliranin}
Nais ng mga mananaliksik na malaman sa pamamagitan ng pagsasagawa ng pagsusuring
ito ang tamang wikang gagamitin sa mga piling asignatura.

Ito ang mga tanong na nais naming masagot sa pagsagawa ng pag-aaral na ito:

\begin{enumerate}[parsep=\baselineskip,itemsep=-0.5\baselineskip]
\item Anong wika ang kanaisnais na gamitin sa mga piling asignatura:
  \begin{enumerate}[nosep]
  \item Mathematics;
  \item Religious Education;
  \item Social Science;
  \item Chem; at
  \item Personal Development?
  \end{enumerate}
\item Nakakaapekto ba ang unang wikang kanilang natutunan sa kanilang sagot?
\item Nakakaapekto ba ang wikang sila'y komportable sa kanilang sagot?
\end{enumerate}

\section{Kahalagahan ng Pag-aaral}
Layunin ng pag-aaral na ito na malaman kung ang pinaniniwalaang pinakaepektibong
wikang ginagamit sa pagtuturo (wikang Bisaya) ay talaga nga bang mabisa sa
pagkatuto ng mga mag-aaral ng USC-TC SHS. Ito’y malalaman namin sa pamamagitan
ng pagtutukoy kung saang wika mas komportable ang bawat estudyante at
samakatwid, pagtutukoy na rin kung saang wika mas may maraming natutunan ang
isang mag-aaral. Saklaw ng aming pag-aaral ang mga nais ipagamit na wika ng mga
mag-aaral ng USC-TC SHS sa pagtuturo sa kanila ng mag-iba-ibang mga piling
asignatura (Mathematics, Religious Education, Social Science, Chemistry at
Personal Development). Ang pananaliksik na ito ay sadyang makabuluhan sapagkat
ang impormasyong makakalap namin rito ay maaaring gamitin ng mga guro bilang
gabay sa kanilang pagtuturo sa hinaharap upang makamit ang mas masagana’t mas
mabisang interaksyon at pakikipaghalubilo nila sa kanilang mga tinuturuan. At sa
kahuli-hulihan, layon ng pag-aaral na ito na lalong mapabuti at mapahusay ang
pag-aaral ng isang estudyante dito sa USC-TC, upang masiguro ang garantiya ng
kanilang kinabukasan.

\pagebreak
\section {Saklaw at Limitasyon ng Pag-aaral}
Ang pagsusuring ito ay nalilimita lamang sa mga USC SHS STEM students. Nais
naming malaman kung anong wika ang tamang gamitin sa mga \emph{piling
  asignatura}.

Walang kasiguraduhan na ito ay pareho rin sa ibang mga paaralan. Hindi layunin
ng pagsusuring ito ang magkaroon ng iisang wika lamang sa lahat ng mga
asignatura.

\section{Depinisyon ng mga Terminolohiya}

\chapter{Konseptuwal o Toretikal na Balangkas at mga Kauganayan na Pag-aaral ng Literatura}
\section {Konseptuwal o Toretikal na Balangkas}
\pagebreak
\section {Mga Kaugnayan na Pag-aaral at Literatura}
Ayon kay Santrock (1990), sa Estados Unidos, maraming mga batang imigranteng
pumunta mula sa pagiging monolingual sa kanilang wikang sa bilingual sa wikang
iyon at sa Ingles, lamang sa mga end up monolingual mga nagsasalita ng Ingles.
Ang tawag dito ay \emph{subtractive bilingualism}, at ito ay maaaring magkaroon
ng negatibong epekto sa mga bata, na madalas mong ikahiya kanilang wikang.

Ayon kay Orlich, Harder, Trevisan, Brown, Callahan (2009), mahalaga na matuto ang
wika dahil ito ay hindi lang ginagamit upang makipaghalubilo sa ibang tao sa
paaralan kundi ito rin ay importante upang maka-angat sa \emph{social} at
\emph{economic status}; ang pagkatuto ng common o standard na wika ng isang
bansa ay lubos na mahalaga para sa pag-unlad ng econmic at social status ng
isang dayuhan.

Ayon pa rin kay Orlich, et al. (2009), ang ating edukasyonal na polisiya ay
dapat maghikayat ng \emph{inclusion} at hindi ng \emph{exclusion}. Ang
paghikayat ng \emph{multicultural education} sa paaralan ay hindi mahirap at
hindi rin ito nakapapa-walang-bisa sa mga kasanayang prosa ng \emph{curriculum}.

\chapter{Disenyo at Paraan ng Pananaliksik}
\section{Paraan ng Pananliksik}
\section{Mga Respondente}
\section{Instrumento sa Paglakap ng Datos}
Ang ginamit ng mga mananlikisik ay ang \emph{survey questionnaire} dahil ito ay
masmabilis at mas madali sa pagkuha ng impormasyon lalong lalo na kapag malaki
ang \emph{sample size}. Mas madali rin makakuha ng \emph{statistically
  significant} na mga resulta kaysa sa ibang mga paraan ng paglakap ng
impormasyon. Ang survey ay epektibo sa pagsusuri ng maramihang mga
\emph{variables}. Pinaka-ideyal ang \emph{survey questionnaire } dahil
nakokontrol ang mga \emph{stimulus}. Na-\emph{stastandardize} rin ang mga
depinisyon na ginagamit ng mga respondente. Kaya, may masmataas na presisyon sa
paglakap ng datos.

\section{Pagbuo ng Talatanungan}
\section{Paglalapat Estatika}

\chapter{Paglalahad, Pagsusuri at Interpretasyon ng mga Datos}
\chapter{Paglalagom, Kongklusyon at Rekomendasyon}
\section{Lagom}
\section{Konklusyon}
\section{Rekomendasyon}


%Maspinahahalagahan namin ang sagot na nasa choices kaysa sa ``Iba.''

% Ang mga respondente ay ipinili gamit ang \emph{Simple Random Sampling}\footnote{Simple Random Sampling -- Isang paraan ng pagpili na parang lottery}.
\end {document}
%%% Local Variables: 
%%% TeX-engine: xetex
%%% End: 