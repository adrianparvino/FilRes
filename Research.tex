\documentclass [british, 11pt] {report}
\usepackage [margin=1in, left=1.5in, letterpaper]{geometry}


% \usepackage[titles]{tocloft}
\usepackage {titlesec}

\usepackage{standalone}
\usepackage{calc}
\usepackage{tikz}

\usepackage {tikz}
\usepackage {graphicx}
\graphicspath {{.}}

\usepackage{babel}

\usepackage[style=apa,backend=biber]{biblatex}
\addbibresource{Research.bib}
\DeclareLanguageMapping{british}{british-apa}
% \renewcommand\bibname{LISTAHAN NG SANGGUNIAN}

\usepackage{pgfplots}
\pgfplotsset{compat=1.13}

\linespread {1.6}
\usepackage {fontspec}
%\setmainfont {Tahoma}
\setmainfont [ItalicFont={Tahoma Faux Italic}]{Tahoma}

\usepackage{enumitem}
\usepackage{titlesec}
\usepackage[section=section]{glossaries}

\makeglossaries
\glsaddkey*  
{verb}% key  
{} % default value  
{\glsentryverb}% command analogous to \glsentrytext  
{\Glsentryverb}% command analogous to \Glsentrytext  
{\glsverb}% command analogous to \glstext  
{\Glsverb}% command analogous to \Glstext  
{\GLSverb}% command analogous to \GLStext  
\newglossaryentry{student}
{
  name = {USC SHS STEM},
  description = {Mga mag-aaral na naka-enroll sa Senior High School sa
paaralang University of San Carlos}
}

\newglossaryentry{cebuano}
{
  name = {Wikang Cebuano},
  description = {Likas na wika ng Cebu, Visayas, Philippines}
}

\newglossaryentry{srs}
{
  name = {simple random sampling},
  description = {isang paraan ng pagpili na parang lottery}
}

\newglossaryentry{-bilang}
{
  name = {subtractive bilingualism},
  description = {pagbawas ng paggamit ng inang wika ng imigranteng bata hanggang
  sila ay maggingng \emph{monolingual} sa salitang Ingles na lamang}
}

\newglossaryentry{satisfaction}
{
  name = {satisfaction},
  description = {ang pagkasang-ayon ng tao sa isang wika},
  verb = {satsfied}
}

\newglossaryentry{satisfactionfull}
{
  name = {full satisfaction},
  description = {nakakamit sa pagpili ng wika na pareho sa wikang pinili ng repsondente},
  verb = {fully satisfied},
  sort = {satisfaction, full}
}

\begin {document}
\documentclass {article}
\usepackage {fontspec}
\setmainfont [ItalicFont={Tahoma Faux Italic}]{Tahoma}
\begin {document}
\include {Common}
\begin {titlepage}
  \centering
	{\huge \researchtitle

  }
  \vfill
  {\LARGE Isang Sulating Pananaliksik na Iniharap kay \teacher ng

    \uppercase{\school}

  }
  \vspace{1cm}
  {\LARGE Bilang Bahagi ng Pagpatupad sa Pangangailangan ng Kursong
    Filipino 11, Pagbasa at Pagsuri ng Iba't ibang Teksto Tungo sa Pananaliksik

  }
  \vfill
  {\Large\itshape \authors

  }
\end {titlepage}
\end {document}
%%% Local Variables: 
%%% TeX-engine: luatex 
%%% End: 
\newcommand\discard[1]{}
\parindent=0.5in
\parskip=\baselineskip
\titleformat{name=\chapter}[display]
{\bfseries}
{\centering KABANATA \Roman{chapter}}
{-\parskip}{\centering\uppercase}

\titleformat{name=\section}[display]
{\bfseries}
{}
{-\parskip}{}

\titlespacing*{name=\chapter}{0pt}{-2.0\parskip}{-\parskip}
\titlespacing*{name=\section}{0pt}{0pt}{0pt}
\parskip=0em
\tableofcontents
\parskip=\baselineskip
\chapter {Ang Suliranin at ang Kaligiran Nito}
\section {Rasyunal}
Ang Pilipinas ay may mayamang kultura. Ito ay resulta ng pagkakasakop ng bansa
nuon ng mga Kastila, Amerikano at mga Hapones. Ang wikang Filipino ay napalooban
ng mga halo-halong wika ng mga dayuhan. Pero dahil sa pagiging kapuluan ng
Pilipinas, ito ay nagdulot ng kaunting pagkakaiba-iba ng mga wika ng iba't ibang
pulo ng bansa.

Isa sa mga problema ng mga estudyante ay ang hindi pagkakaintindi sa leson dahil
sa kahirapan sa pag-intindi sa wika. Layunin ng pagsusuring ito na isaayos ito.
Pakay namin na maintindihan ng lahat ng mga estudyante ang mga leksiyong
itinituro ng kanilang mga guro sa kanila sa pamamagitan ng pagpapalitaw kung
anu-ano ang mga wastong gamiting wika sa mga piling asignatura. Gusto naming
solusyonan ang problemang ito dahil bilang mga kapwa mag-aaral, kami rin ay
minsan na ring nakaranas nito.

Hindi lang ito nakabubuti sa mga estudyante. Ito rin ay makapapatunay na ito ay
mapapakinabangan ng mga guro. Hindi na palaging paulit-ulit magturo ang titser
sa isang paksa kapag maigi nang naiintindihan ng kanyang mga tinuturuan ang
leksiyon. Ang daloy ng interaksyon ng guro sa kaniyang mga estudyante ay bubuti
at aayos na siya na ma'y siguradong ikalulugod ng dalawang panig.
\pagebreak
\section {Paglalahad ng Suliranin}
Nais ng mga mananaliksik na malaman sa pamamagitan ng pagsasagawa ng pagsusuring
ito ang tamang wikang gagamitin sa mga piling asignatura.

Ito ang mga tanong na nais naming masagot sa pagsagawa ng pag-aaral na ito:
\begin{enumerate}[parsep=0.5\parskip,topsep=-0.5\parskip]
\item Anong wika ang kanaisnais na gamitin sa mga piling asignatura:
  \begin{enumerate}[nosep]
  \item Mathematics;
  \item Religious Education;
  \item Social Science;
  \item Chem; at
  \item Personal Development?
  \end{enumerate}
\item Nakakaapekto ba ang unang wikang kanilang natutunan sa kanilang sagot?
\item Nakakaapekto ba ang wikang sila'y komportable sa kanilang sagot?
\end{enumerate}


\section{Kahalagahan ng Pag-aaral}
Layunin ng pag-aaral na ito na malaman kung ang pinaniniwalaang pinakaepektibong
wikang ginagamit sa pagtuturo (\gls{cebuano}) ay talaga nga bang mabisa sa
pagkatuto ng mga mag-aaral ng USC-TC SHS. Ito'y malalaman namin sa pamamagitan
ng pagtutukoy kung saang wika mas komportable ang bawat estudyante at
samakatwid, pagtutukoy na rin kung saang wika mas may maraming natutunan ang
isang mag-aaral. Saklaw ng aming pag-aaral ang mga nais ipagamit na wika ng mga
mag-aaral ng USC-TC SHS sa pagtuturo sa kanila ng mag-iba-ibang mga piling
asignatura (Mathematics, Religious Education, Social Science, Chemistry at
Personal Development). Ang pananaliksik na ito ay sadyang makabuluhan sapagkat
ang impormasyong makakalap namin rito ay maaaring gamitin ng mga guro bilang
gabay sa kanilang pagtuturo sa hinaharap upang makamit ang mas masagana't mas
mabisang interaksyon at pakikipaghalubilo nila sa kanilang mga tinuturuan. At sa
kahuli-hulihan, layon ng pag-aaral na ito na lalong mapabuti at mapahusay ang
pag-aaral ng isang estudyante dito sa USC-TC, upang masiguro ang garantiya ng
kanilang kinabukasan.

\pagebreak
\section {Saklaw at Limitasyon ng Pag-aaral}
Ang pagsusuring ito ay nalilimita lamang sa mga \glspl{student}. Nais
naming malaman kung anu-anong mga wika ang angkop gamitin sa mga \emph{piling
  asignatura}.

Walang kasiguraduhan na ang impormasyong makakalap sa pagsusuring ito ay hindi
nag-iiba sa ibang paaralan. Hindi layunin ng pagsusuring ito ang magkaroon ng
iisang wika lamang sa lahat ng mga asignatura.

Hindi rin masisigurado ng pag-aaral na ito na ang mga mairerekomenda wikang
ipagamit sa bawat asignatura na kasaklaw rito ay ang tiyak na pinakamabisang
gamitin sa asignaturang iyon sapagkat opinyon lamang ng mga estudyante ang aming
gamit sa pagsusuri upang makamit ang impormasyong ito.

\printglossary

\chapter{Konseptuwal o Toretikal na Balangkas at mga Kauganayan na Pag-aaral ng Literatura}
\section {Konseptuwal o Toretikal na Balangkas}
Ang konsepto ng Bilingualism ay itinalakay na sa mga mag-iba-ibang perspektibong
toretikal. Ngunit dahil sa likas na pagka-diverse ng paksang ito, ang iisang
teoryang pangunibersal na mag-uugnay sa iba-ibang mga perspektibo patungkol rito
ay malabong makamit at kung meron man, hindi ito ganap na maituturing na
eksakto.

Karaniwan sa ating mga Pilipino, kung hindi lahat, ay may multilingual na dila.
Ang sitwasyon ng wika sa Pilipinas, isang bansang sinasaklaw ng higit-kumulang
pitong libong mga pulo, ay pawang complicado sa heograpikong diwa man o sa
makasaysayang pag-unawa. Ayon kay \textcite{KAWAHARA}, mayroong walong pangunahing
wika sa bansa--Ilokano, Pampangan, Pangasinan, Tagalog, Bicol, Hiligaynon, Waray
at Cebuano--at inilahad na pagkatapos pa ng walong pangunahing mga wikang ito ay
iba pang sari-saring mga wikang sinasalita ng ilang daang libong mga Pilipino.
At sa bingit pa ng mga nagsasaring wikang ito ay ang mga wikang sinasalita ng
mga minority races na tinutukoy bilang mga cultural minorities (p. 67).

May maraming teoryang naglalahad sa importansya ng paggamit ng inang wika sa mga
elementaryang baitang. Ayon sa isang pananaliksik ni \cite{BROCKUTNE}, mas
madaling matututo ang mga estudyante kapag inang wika ang gagamitin bilang
medyum sa pagtuturo. Ito rin ay ang pinakamaiging gamitin bilang tulay sa
pagkatuto ng mga pangunahing wika ng pagtuturo [eg. Ingles].
\pagebreak
\section {Mga Kaugnayan na Pag-aaral at Literatura}
Sa Estados Unidos, maraming mga batang imigranteng nanggaling sa paggiging
monolingual sa kanilang inang wika. Pagkalaunan, lumipas ang ilang taon, sila ay
naging bilingual sa kanilang inang wika at sa wikang Ingles, na siyang
pambansang wika ng Estados Unidos. Ngunit sa pagdaan pa ng ilang taong
paninirahan sa Estados Unidos ang mga batang ito ay nakaangkop na sa kultura ng
bansa at upang sila ay hindi ituturing na outcast sa ibang mga bata, magkakaroon
ang bata ng pagkakagawing gamitin ang Ingles bilang wika, ang itinuturing na
``normal'' sa bansang iyon. At sanhi ng labis na pagbawas ng paggamit ng inang
wika ng mga imigranteng bata, ang wikang ito ay unti-unting maglalaho hanggang
ang bata ay magiging monolingual sa salitang Ingles na lamang. Ito ang tinatawag
na \emph{\gls{-bilang}} \parencite{SANTROCK}.

Ang kaalamang ito ay importante sa aming kasalukuyang pananaliksik. Ang
pag-aaral ni \textcite{SANTROCK} ay nagpapahiwatig na hindi dahil iyan ang unang
wikang natutunan mo, iyan na ang wikang magagamit mo sa hinaharap na mga gawain.
Malaking impluwensiya ang kultura sa iyong pag-uugali't pananalita. Sa pag-aaral
na ito malalaman natin kung ano nga ba ang mas nakaaapekto sa mas matimbang na
kagustuhan ng karaniwan, ito ba ay ang kanilang inang wika o ang wikang
kadalasang ginagamit sa kanilang paaralan nang sila ay lumaki.

Sa ngayon, ang pinakamadalas gamitin sa pagtuturo ay ang wikang Ingles. Ayon kay
\textcite{KOO} ito ay dahil ang Ingles ang wikang angkop gamitin para sa mga paksang
pang-ekonomiyang pagsulong, mga pang-akademikong gawain, at globalisasyon (22).
Ipinahihiwatig ni \textcite{KOO} rito na mahalaga ang pag-aaral ng salitang Ingles,
ngunit wala siyang binanggit na mas epektibong medyum ng pagtuturo ang Ingles.

Ang Ingles ay madalas ginagamit sa mga pormal na pagtitipon at mga institution.
Giit sa pag-aaral ni \textcite{BORLONGAN}, hindi magiging epektibo ang pagtuturo
kung ang gamit na wika ay ang inang wika lamang. Dapat may halo itong konting
Ingles upang maunawaan ng estudyante nang maigi ang leksiyon. Pero ganunpaman,
ang pag-aaral na ito nina \textcite{KOO,BORLONGAN} ay isinagawa sa lungsod
ng Manila, kung saan ang nangingibabaw na mga wika ay ang Ingles at Filipino
habang ang kasalukuyang pananaliksik na ito ay isinagawa rito sa Cebu lamang
kung saan ang mga nangingibabaw na wika ay Bisaya at Ingles.

Importanteng importante na malaman natin ang wastong wikang gagamitin dahil ang
wika ay hindi lang ginagamit upang makipaghalubilo sa ibang tao sa paaralan
kundi ito rin ay importante upang maka-angat sa social at economic
status \parencite{ORLICH}


Ayon pa rin kay \textcite{ORLICH}, ang ating mga polisyang edukasyonal ay
dapat maghikayat ng inclusion at hindi ng exclusion. Ang paghikayat ng
multicultural education sa paaralan ay hindi mahirap at hindi rin ito
nakapapa-walang-bisa sa mga kasanayang prosa ng curriculum. Gusto ng lahat na
maani ang angking potensyal ng bawat estudyante kaya mahalagang malaman ang
pinakamabisang wikang gagamitin sa pag-aaral ng bawat larangan o asignatura.
\chapter{Disenyo at Paraan ng Pananaliksik}
\section{Paraan ng Pananliksik}
Ang pananliksik namin ay \emph{quantitative}. Ito ay mahahalintulad sa isang 1--4 scale.
Ang 1 ay pinagpalit sa sagot na ``Bisaya'' habang ang 4 naman ay ang pinagpalit
ng sagot na ``Ingles.''
\section{Mga Respondente}
Ang mga respondente ay ang mga \glspl{student}. Pinili namin ang mga
\gls{student} dahil ang Senior High School ay ang  pinaka-experimental na
curriculum at ang kanilang opinyon ay lubos na mahalaga. Ang mga respondente ay
pinipili sa pamamagitan ng \Gls{srs}.
\section{Instrumento sa Paglakap ng Datos}
Ang ginamit ng mga mananlikisik ay ang \emph{survey questionnaire} dahil ito ay
masmabilis at mas madali sa pagkuha ng impormasyon lalong lalo na kapag malaki
ang \emph{sample size}. Mas madali rin makakuha ng \emph{statistically
  significant} na mga resulta kaysa sa ibang mga paraan ng paglakap ng
impormasyon. Ang survey ay epektibo sa pagsusuri ng maramihang mga
\emph{variables}. Pinaka-ideyal ang \emph{survey questionnaire } dahil
nakokontrol ang mga \emph{stimulus}. Na-\emph{stastandardize} rin ang mga
depinisyon na ginagamit ng mga respondente. Kaya, may masmataas na presisyon sa
paglakap ng datos.

\section{Pagbuo ng Talatanungan}
Napagsunduan naming gawing simple lamang ang aming talatanungan upang mapadali
at mapabilis ang aming pagkolekta ng data. Isa pang benepisyo ng ganitong uri ng
pag-organisa ay ang pagkakombenyente nito para sa aming mga respondente.

Ang format ng talatanungan ay pina-horizontal na tseklist; sa kaliwang bahagi ng
talatanungan namin inilagay ang mga tanong at sa kanang bahagi ng papel naman ay
limang kulumna, apat ay mga munting kahon kung saan ang nailagay sa ulohan ng
bawat hanay ay ``Bisaya,'' ``Ingles,'' ``Bisaya na may halong Ingles'' at
``Ingles na may halong bisaya,'' at sa ikalima ay isang kolumnang hanay ng mga
munting salungguhit na may titulong ``Iba.'' Dito maaaring ilagay ng mga
respondente ang kanilang sagot kung sakali mang ang kanilang nais isagot sa
tanong ay wala sa mga pagpipilian.

Ang unang tanong ay ``Ano ang unang wikang iyong natutunan?'' Dito natin
malalaman kung ano ang kanilang inang wika. Ang wikang pinakaunang naituro sa
kanila.
  
Ikalawang tanong ay ``Ako ay mas komportableng magsalita sa wikang\ldots''
Malalaman natin sa katanungang ito kung sa anong wika mas bihasa ang
respondente.

Ang ikatlo at ikaapat na tanong ay ``Mas marami akong natutunan kung ang gamit
ay\ldots'' at ``Ito ang pinakamagaling gamiting wika sa klase.'' Ang mga
katanungang ito ang magsasabi kung anong pinakaangkop na wikang nais gamitin at
ipagamit ng mga estudyante sa klase. Malalaman rin dito ang wika kung saan sila
mas natututo sa klase sa pangkalahatang diwa. Ito'y maaari nating ihambing sa
impormasyong ating makakalap sa una at pangalawang tanong.

Ang ikalima hanggang sa ikasiyam at huling tanong ay pawang mga tanunging
nagtatanong kung anong gustong gamitin at ipagamit ng mga estudyante sa
pagtuturo't pagkatuto sa sari-saring mga subject.
\section{Paglalapat Estatika}
%% TODO: Change happy/happpiness.
Kukunin namin ang \emph{Mode} at \emph{Mean} ng mga nalakap na datos. Lalong
mahalaga ang pagkuha ng \emph{Mean} dahil dito natin makamit ang
\emph{\gls{satisfactionfull}} ng lahat ng tao. At ang \emph{Mode} naman ay mahalaga
dahil dito natin mamaximize ang karamihan ng mga taong
\emph{\glsverb{satisfaction}}.

\chapter{Paglalahad, Pagsusuri at Interpretasyon ng mga Datos}
\begin{tikzpicture}
  \begin{axis}[
    ybar stacked,
    symbolic x coords={Unang wika, Comfortable, Mas maraming natutunan, Magaling, Math, Reed, Socsci, Chem, Pdev},
    xtick=data,
    nodes near coords,
    width=\textwidth,
    height=0.33\textheight,
    x tick label style={rotate=45,anchor=east},
    legend style={at={(0.5,-0.3)},anchor=north}]
    \addplot+ table [x=Question, y=Bisaya, col sep=comma]{Data.csv};
    \addplot+ table [x=Question, y=Bisaya na may halong Ingles, col sep=comma]{Data.csv};
    \addplot+ table [x=Question, y=Ingles na may halong Bisaya, col sep=comma]{Data.csv};
    \addplot+ table [x=Question, y=Ingles, col sep=comma]{Data.csv};
    \addplot+ table [x=Question, y=Iba, col sep=comma]{Data.csv};
    \legend{Bisaya, Bisaya na may halong Ingles, Ingles na may halong Bisaya, Ingles, Iba}
  \end{axis}
\end{tikzpicture}

\noindent
Ang mga modes ng Unang Wika at Comfortable ay Bisaya ngunit ang mga modes naman
ng iba ay ang Ingles. Kahit na pinakamaraming tao ay Unang Wika at Comfortable
sa Bisaya, masmarami ang gustong maggamit ng Ingles sa pagkatuto. Kapag gusto
i-maximize ang dami ng tao na \glsverb{satisfactionfull}, kailangan gamitin ang
Ingles sa pagkatuto.

Ang mean naman ay lahat nagpupunterya sa Ingles na may halong Bisaya. Kapag
ang pakay natin ay mamaximize ang kabuuang \gls{satisfaction}, ito ang
pinakaepektibong solusyon.


\chapter{Paglalagom, Kongklusyon at Rekomendasyon}
\section{Lagom}
Ang pangunahing wika ng 84.81\% ng mga estudyanteng Senior High sa USC-TC ay
Bisaya. 37\% ng mga estudyanteng ito ay may halong kaunting Ingles ang kanilag
pagkatuto ng inang wika. Sa kabilang dako naman, mayroon lamang 15.19 \% na mga
mag-aaral kung saan Ingles o ibang wika ang pangunahin nilang wikang natutunan.
Sa kasalukuyan, nangunguna pa rin ang wikang Bisaya kung ang pag-uusapan ay kung
anong wika mas komportableng gamitin ng mga estudyante (77.21 \%.) 32\% ng mga
estudyanteng saklaw nito ay mas komportable kung itoy gamitin kasabay ng halong
Ingles. Pero mayroon pa ring 21.25\% na estudyanteng pipiliing gumamit ng Ingles
sa pang-araw-araw na komunikasyon.

Sa pang-akademikong pananaw naman, tila mas nangingibabaw ang Ingles sa datos.
Ayon sa 53.16\% ng populasyon, mas marami silang matututunan kung Ingles ang
gagamiting wika sa pagtuturo sa kanila. 20\% ng populasyon ang nagsasabing
Bisaya ang mas epektibong gamiting sa pag-aaral at pagkatuto. 17\% ng mga
estudyante ay nagsasabing mas mabisa kung ang gagamiting wika ay Bisaya ngunit
ito'y paghaluan ng kaunting Ingles. Ang tingin naman ng natitirang 10 \% ay
dapat ang gamiting wika sa pag-aaral ay Ingles, ngunit ito'y maaaring paghaluan
ng kaunting Bisaya.

Sa mga asignatura naman, pare-parehong ninanais ng kabuuan ng populasyon na
ituro ang Math (58.23\%), REED(45.57\%), SocSci (51.90\%), Chem(50.63\%), at
Pdev (46.84\%) sa wikang Ingles. Kung d naman Ingles, gusto ng kabuuan ng
populasyon na ituro ang mga larangang ito sa wikang Bisaya na may halong
kaunting wikang Ingles.
\section{Konklusyon}
Kahit Bisaya pa man ang kadalasang pangunahing wika ng mga estudyante, at kahit
ang wikang ito ang wikang mas komportableng gamitin nga mga mag-aaral, masasabi
namin, sa pamamagitan ng pagsusuri ng aming research, na ang mas epektibong
gamiting wika sa klase ay ang wikang Ingles. Ito ay naipahihiwatig sa nakalap
naming impormasyon at datos. Ang katunayang mas ninanais ng mga estudyante ang
matuto at mag-aral sa wikang Ingles ay isa nang malinaw na palatandaan na mas
marami silang nauunawan kung ito ang ginagamit na linggwahe bilang medyum ng
pagpapakaloob ng impormasyon.
\section{Rekomendasyon}
\begin{enumerate}
\item Palakihin ang saklaw ng pag-aaral. Saliksikin ang buong USC-TC. Kumuha ng
  mga sample mula sa bawat kurso o departmento kung possible.
\item Isali sa mga katanungan ang wikang ginagamit sa pakikipaghalubilo sa loob
  ng bahay, sa labas at sa loob ng unibersidad. Ihambing ang mga makakalap na
  impormasyon.
\item Magsigawa ng karagdagang pananaliksik kung bakit mas ninanais ng mga
  mag-aaral na gamitin ang wikang Ingles sa mga pang-akademikong gawain.
\end{enumerate}
 
\printbibliography[title = {LISTAHAN NG SANGGUNIAN}]
\IfFileExists{Biography.tex}{\input{Biography}}{}
\end {document}
%%% Local Variables: 
%%% TeX-engine: xetex
%%% End: 