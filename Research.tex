\documentclass [11pt] {report}
\usepackage [margin=1in, left=1.5in, letterpaper]{geometry}

% \usepackage[titles]{tocloft}

\usepackage {titlesec}

\usepackage {luatex85}

\usepackage{standalone}
\usepackage{calc}

\linespread {1.3}
\usepackage {fontspec}
%\setmainfont {Tahoma}
\setmainfont [ItalicFont={Tahoma Faux Italic}]{Tahoma}

\usepackage{enumitem}
\usepackage{titlesec}

% \renewcommand{\cftchapfont}{%
%   \normalfont\bfseries
% }

\begin {document}
\documentclass {article}
\usepackage {fontspec}
\setmainfont [ItalicFont={Tahoma Faux Italic}]{Tahoma}
\begin {document}
\include {Common}
\begin {titlepage}
  \centering
	{\huge \researchtitle

  }
  \vfill
  {\LARGE Isang Sulating Pananaliksik na Iniharap kay \teacher ng

    \uppercase{\school}

  }
  \vspace{1cm}
  {\LARGE Bilang Bahagi ng Pagpatupad sa Pangangailangan ng Kursong
    Filipino 11, Pagbasa at Pagsuri ng Iba't ibang Teksto Tungo sa Pananaliksik

  }
  \vfill
  {\Large\itshape \authors

  }
\end {titlepage}
\end {document}
%%% Local Variables: 
%%% TeX-engine: luatex 
%%% End: 
\newcommand\discard[1]{}
% \RedeclareSectionCommand[
%   font=\normalfont\bfseries,
%   tocentrynumberformat=\discard,
%   beforeskip=0pt,
%   innerskip=1pt,
%   afterskip=1sp]{chapter}
% \RedeclareSectionCommand[
%   font=\normalfont\bfseries,
%   tocentrynumberformat=\discard,
%   beforeskip=0pt,
%   afterskip=1sp]{section}
\parindent=0.5in
\parskip=\baselineskip
\titleformat{name=\chapter}[display]
{\bfseries}
{\centering KABANATA \Roman{chapter}}
{-\parskip}{\centering\uppercase}

\titleformat{name=\section}[display]
{\bfseries}
{}
{-\parskip}{}

\titlespacing*{name=\chapter}{0pt}{-2.0\parskip}{-\parskip}
\titlespacing*{name=\section}{0pt}{0pt}{0pt}
%\renewcommand*{\chapterformat}{\centering Kabanata \thechapter:\enskip}
%\renewcommand*{\sectionformat}{}
\chapter {Ang Suliranin at ang Kaligiran Nito}
\section {Rasyunal}
Ang Pilipinas ay may mayamang kultura. Ito ay resulta ng pagkakasakop ng bansa
nuon ng mga Kastila, Amerikano at mga Hapones. Ang wikang Filipino ay napalooban
ng mga halo-halong wika ng mga dayuhan. Pero dahil sa pagiging kapuluan ng
Pilipinas, ito ay nagdulot ng kaunting pagkakaiba-iba ng mga wika ng iba’t ibang
pulo ng bansa.

Isa sa mga problema ng mga estudyante ay ang hindi pagkakaintindi sa leson dahil
sa kahirapan sa pag-intindi sa wika. Layunin ng pagsusuring ito na isaayos ito.
Pakay namin na maintindihan ng lahat ng mga estudyante ang mga leksiyong
itinituro ng kanilang mga guro sa kanila sa pamamagitan ng pagpapalitaw kung
anu-ano ang mga wastong gamiting wika sa mga piling asignatura. Gusto naming
solusyonan ang problemang ito dahil bilang mga kapwa mag-aaral, kami rin ay
minsan na ring nakaranas nito.

Hindi lng ito nakabubuti sa mga estudyante. Ito rin ay makapapatunay na ito ri’y
mapapakinabangan ng mga guro. Hindi na palaging paulit-ulit magturo ang titser
sa isang paksa kapag maigi nang naiintindihan ng kanyang mga tinuturuan ang
leksiyon. Ang daloy ng interaksyon ng guro sa kaniyang mga estudyante ay bubuti
at aayos na siya na ma'y siguradong ikalulugod ng dalawang panig.
\pagebreak
\section {Paglalahad ng Suliranin}
Nais ng mga mananaliksik na malaman sa pamamagitan ng pagsasagawa ng pagsusuring
ito ang tamang wikang gagamitin sa mga piling asignatura.

Ito ang mga tanong na nais naming masagot sa pagsagawa ng pag-aaral na ito:

\begin{enumerate}[parsep=\baselineskip,itemsep=-0.5\baselineskip]
\item Anong wika ang kanaisnais na gamitin sa mga piling asignatura:
  \begin{enumerate}[nosep]
  \item Mathematics;
  \item Religious Education;
  \item Social Science;
  \item Chem; at
  \item Personal Development?
  \end{enumerate}
\item Nakakaapekto ba ang unang wikang kanilang natutunan sa kanilang sagot?
\item Nakakaapekto ba ang wikang sila'y komportable sa kanilang sagot?
\end{enumerate}

\section{Kahalagahan ng Pag-aaral}
Layunin ng pag-aaral na ito na malaman kung ang pinaniniwalaang pinakaepektibong
wikang ginagamit sa pagtuturo (wikang Bisaya) ay talaga nga bang mabisa sa
pagkatuto ng mga mag-aaral ng USC-TC SHS. Ito’y malalaman namin sa pamamagitan
ng pagtutukoy kung saang wika mas komportable ang bawat estudyante at
samakatwid, pagtutukoy na rin kung saang wika mas may maraming natutunan ang
isang mag-aaral. Saklaw ng aming pag-aaral ang mga nais ipagamit na wika ng mga
mag-aaral ng USC-TC SHS sa pagtuturo sa kanila ng mag-iba-ibang mga piling
asignatura (Mathematics, Religious Education, Social Science, Chemistry at
Personal Development). Ang pananaliksik na ito ay sadyang makabuluhan sapagkat
ang impormasyong makakalap namin rito ay maaaring gamitin ng mga guro bilang
gabay sa kanilang pagtuturo sa hinaharap upang makamit ang mas masagana’t mas
mabisang interaksyon at pakikipaghalubilo nila sa kanilang mga tinuturuan. At sa
kahuli-hulihan, layon ng pag-aaral na ito na lalong mapabuti at mapahusay ang
pag-aaral ng isang estudyante dito sa USC-TC, upang masiguro ang garantiya ng
kanilang kinabukasan.

\pagebreak
\section {Saklaw at Limitasyon ng Pag-aaral}
Ang pagsusuring ito ay nalilimita lamang sa mga USC SHS STEM students. Nais
naming malaman kung anu-anong mga wika ang angkop gamitin sa mga \emph{piling
  asignatura}.

Walang kasiguraduhan na ang impormasyong makakalap sa pagsusuring ito ay hindi
nag-iiba sa ibang paaralan. Hindi layunin ng pagsusuring ito ang magkaroon ng
iisang wika lamang sa lahat ng mga asignatura.

Hindi rin masisigurado ng pag-aaral na ito na ang mga mairerekomenda wikang
ipagamit sa bawat asignatura na kasaklaw rito ay ang tiyak na pinakamabisang
gamitin sa asignaturang iyon sapagkat opinyon lamang ng mga estudyante ang aming
gamit sa pagsusuri upang makamit ang impormasyong ito.
\section{Depinisyon ng mga Terminolohiya}

\chapter{Konseptuwal o Toretikal na Balangkas at mga Kauganayan na Pag-aaral ng Literatura}
\section {Konseptuwal o Toretikal na Balangkas}
\pagebreak
\section {Mga Kaugnayan na Pag-aaral at Literatura}
Ayon kay Santrock (1990), sa Estados Unidos, maraming mga batang imigranteng
pumunta mula sa pagiging monolingual sa kanilang wikang sa bilingual sa wikang
iyon at sa Ingles, lamang sa mga end up monolingual mga nagsasalita ng Ingles.
Ang tawag dito ay \emph{subtractive bilingualism}, at ito ay maaaring magkaroon
ng negatibong epekto sa mga bata, na madalas mong ikahiya kanilang wikang.

Ayon kay Orlich, Harder, Trevisan, Brown, Callahan (2009), mahalaga na matuto ang
wika dahil ito ay hindi lang ginagamit upang makipaghalubilo sa ibang tao sa
paaralan kundi ito rin ay importante upang maka-angat sa \emph{social} at
\emph{economic status}; ang pagkatuto ng common o standard na wika ng isang
bansa ay lubos na mahalaga para sa pag-unlad ng econmic at social status ng
isang dayuhan.

Ayon pa rin kay Orlich, et al. (2009), ang ating edukasyonal na polisiya ay
dapat maghikayat ng \emph{inclusion} at hindi ng \emph{exclusion}. Ang
paghikayat ng \emph{multicultural education} sa paaralan ay hindi mahirap at
hindi rin ito nakapapa-walang-bisa sa mga kasanayang prosa ng \emph{curriculum}.

\chapter{Disenyo at Paraan ng Pananaliksik}
\section{Paraan ng Pananliksik}
\section{Mga Respondente}
\section{Instrumento sa Paglakap ng Datos}
Ang ginamit ng mga mananlikisik ay ang \emph{survey questionnaire} dahil ito ay
masmabilis at mas madali sa pagkuha ng impormasyon lalong lalo na kapag malaki
ang \emph{sample size}. Mas madali rin makakuha ng \emph{statistically
  significant} na mga resulta kaysa sa ibang mga paraan ng paglakap ng
impormasyon. Ang survey ay epektibo sa pagsusuri ng maramihang mga
\emph{variables}. Pinaka-ideyal ang \emph{survey questionnaire } dahil
nakokontrol ang mga \emph{stimulus}. Na-\emph{stastandardize} rin ang mga
depinisyon na ginagamit ng mga respondente. Kaya, may masmataas na presisyon sa
paglakap ng datos.

\section{Pagbuo ng Talatanungan}
\section{Paglalapat Estatika}

\chapter{Paglalahad, Pagsusuri at Interpretasyon ng mga Datos}
\chapter{Paglalagom, Kongklusyon at Rekomendasyon}
\section{Lagom}
\section{Konklusyon}
\section{Rekomendasyon}


%Maspinahahalagahan namin ang sagot na nasa choices kaysa sa ``Iba.''

% Ang mga respondente ay ipinili gamit ang \emph{Simple Random Sampling}\footnote{Simple Random Sampling -- Isang paraan ng pagpili na parang lottery}.
\end {document}
%%% Local Variables: 
%%% TeX-engine: xetex
%%% End: 